\documentclass{article}

\usepackage[utf8]{inputenc}
\usepackage{minted}
\usepackage{enumitem}
\usepackage{hyperref}
\usepackage{amsfonts}
\usepackage{amsmath}
\usepackage{indentfirst}

\hypersetup{
    colorlinks=true,
    urlcolor=blue
}

\title{18.S097 PS2 (in Swift)}
\author{Jasdev Singh}

\begin{document}

\maketitle

\section*{Acknowledgements.}

\href{https://twitter.com/tomasruizlopez}{Tomás Ruiz-López} helped me get a better sense of Bow’s higher-kinded-type emulation in answering 2a.

\section*{Question 1.}

Rephrased, Question 1 is asking for the cardinality of the objects in the functor category, $[\textbf{Set}, \textbf{3}]$, i.e. $|\textbf{Obj}([\textbf{Set}, \textbf{3}])|$. To start, we have three functors into $\textbf{3}$ at hand $K_1, K_2, K_3 \in \textbf{Obj}([\textbf{Set}, \textbf{3}])$ (borrowing notation from Question 2).

In asking if there are any functors beyond these, we need to make an observation about $\textbf{Set}$’s hom-sets. $\forall S_1, S_2 \in \textbf{Obj}(\textbf{Set}), S_1 \neq \emptyset, S_2 \neq \emptyset: \textrm{Hom}(S_1, S_2) \neq \emptyset$. Home-sets in $\textbf{Set}$ are non-empty for all \textit{non-empty} set pairings. There will always be functions between them.
Now, let's turn our attention to the empty set. Its self-hom-set only contains the identity morphism, $\textrm{id}_{\emptyset}$. And all of its originating hom-sets—$\textrm{Hom}(\emptyset, S)$ for $S \in \textbf{Obj}(\textbf{Set}), S \neq \emptyset$—are empty since there we can’t construct functions from non-empty sets into the empty set.

We can lean on the above to make sure connections (morphisms) aren’t broken across our functors.

Let’s make this more precise.

The remaining functors need to map $\emptyset$ to an object in $\textbf{3}$ that \textit{only} has outbound morphisms. Then, all other objects in $\textbf{Set}$ must be mapped in the usual way. Listing out the steps for our next candidate functor $F_1$,

\begin{enumerate}[label=(\alph*)]

\item Map $\emptyset$ and $\textrm{id}_{\emptyset}$ to $1$ and $\textrm{id}_1$, respectively.

\item Map all $\emptyset$-originating morphisms to $a$.

\item Map all non-empty sets, their identities, and morphisms between non-empty sets to $2$ and $\textrm{id}_2$ (akin to $K_2$).

\end{enumerate}

We have some degrees of freedom in carrying out (a)–(c). That is, we can form another functor $F_2$ by, substituting in $2$ for all mentions of $1$ in (a). $b$ for $a$ in (b). And $3$ for all occurrences of $2$ in (c).

Similarly, to form an $F_3$. Leave (a) as is. $b \circ a$ for $a$ in (b). And $3$ for all occurrences of $2$ in (c).

$K_1, K_2, K_3, F_1, F_2, F_3$ exhaust all possible structure preserving mappings between $\textbf{Set}$ and $\textbf{3}$.

\section*{Question 2.}

\begin{enumerate}[label=(\alph*)]

\item 

\item 
(Assume \href{https://bow-swift.io}{Bow} is imported for all snippets.)

\end{enumerate}

\section*{Question 3.}

\section*{Question 4.}

\section*{Question 5.}

\section*{Question 6.}

\section*{Question 7.}

\section*{Question 8.}

\end{document}
