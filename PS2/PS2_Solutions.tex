\documentclass{article}

\usepackage[utf8]{inputenc}
\usepackage{minted}
\usepackage{enumitem}
\usepackage{hyperref}
\usepackage{amsfonts}
\usepackage{amsmath}
\usepackage{indentfirst}
\usepackage[all]{xy}
\usepackage{mathtools}
\usepackage{amssymb}

\hypersetup{
    colorlinks=true,
    urlcolor=blue
}

\title{18.S097 PS2 (in Swift)}
\author{Jasdev Singh}

\begin{document}

\maketitle

\section*{Acknowledgements.}

\href{https://twitter.com/tomasruizlopez}{Tomás Ruiz-López} helped me get a better sense of Bow’s higher-kinded-type emulation in answering 2b. Also, while learning how to typeset diagrams, I referenced \href{https://twitter.com/mbrandonw}{Brandon Williams}’s \href{https://github.com/mbrandonw/my-math-notes}{graduate school notes}.

\section*{Question 1.}

Rephrased, Question 1 is asking for the cardinality of the objects in the functor category, $[\textbf{Set}, \textbf{3}]$, i.e. $|\textbf{Obj}([\textbf{Set}, \textbf{3}])|$. To start, we have three functors into $\textbf{3}$ at hand $K_1, K_2, K_3 \in \textbf{Obj}([\textbf{Set}, \textbf{3}])$ (borrowing notation from Question 2).

In asking if there are any functors beyond these, we need to make an observation about $\textbf{Set}$’s hom-sets. $\forall S_1, S_2 \in \textbf{Obj}(\textbf{Set}), S_1 \neq \emptyset, S_2 \neq \emptyset: \textrm{Hom}(S_1, S_2) \neq \emptyset$. Home-sets in $\textbf{Set}$ are non-empty for all \textit{non-empty} set pairings. There will always be functions between them.
Now, let's turn our attention to the empty set. Its self-hom-set only contains the identity morphism, $\textrm{id}_{\emptyset}$. And all of its originating hom-sets—$\textrm{Hom}(\emptyset, S)$ for $S \in \textbf{Obj}(\textbf{Set}), S \neq \emptyset$—are empty since there we can’t construct functions from non-empty sets into the empty set.

We can lean on the above to make sure connections (morphisms) aren’t broken across our functors.

Let’s make this more precise.

The remaining functors need to map $\emptyset$ to an object in $\textbf{3}$ that \textit{only} has outbound morphisms. Then, all other objects in $\textbf{Set}$ must be mapped in the usual way. Listing out the steps for our next candidate functor $F_1$,

\begin{enumerate}[label=(\alph*)]

\item Map $\emptyset$ and $\textrm{id}_{\emptyset}$ to $1$ and $\textrm{id}_1$, respectively.

\item Map all $\emptyset$-originating morphisms to $a$.

\item Map all non-empty sets, their identities, and morphisms between non-empty sets to $2$ and $\textrm{id}_2$ (akin to $K_2$).

\end{enumerate}

We have some degrees of freedom in carrying out (a)–(c). That is, we can form another functor $F_2$ by, substituting in $2$ for all mentions of $1$ in (a). $b$ for $a$ in (b). And $3$ for all occurrences of $2$ in (c).

Similarly, to form an $F_3$. Leave (a) as is. $b \circ a$ for $a$ in (b). And $3$ for all occurrences of $2$ in (c).

$K_1, K_2, K_3, F_1, F_2, F_3$ exhaust all possible structure preserving mappings between $\textbf{Set}$ and $\textbf{3}$.

\section*{Question 2.}

\begin{enumerate}[label=(\alph*)]

\item $K_B$ preserves

\begin{enumerate}[label=-]
        \item compositions, since $\forall S_1, S_2, S_3 \in \textbf{Obj}(\textbf{Set}) \textrm{ and } \forall f: S_1 \rightarrow S_2, \forall g: S_2 \rightarrow S_3, K_B(g \circ f) = \textrm{id}_B = \textrm{id}_B \circ \textrm{id}_B = K_B(g) \circ K_B(f)$.
        \item identities, since $\forall S \in \textbf{Obj}(\textbf{Set}), K_B(\textrm{id}_S) = \textrm{id}_B = \textrm{id}_{K_B(S)}$.
    \end{enumerate}

\item Mirroring Bow’s approach to \href{https://bow-swift.io/docs/fp-concepts/higher-kinded-types/}{higher-kinded-type emulation} and with \texttt{boolConst} representing the constant functor on Swift’s \texttt{Bool} type.

\begin{minted}{swift}
class Kind<F, A> {
	init() {}
}

final class ForConst {}
final class ConstPartial<Constant>: Kind<ForConst, Constant> {}
typealias ConstOf<Constant, Tag> = Kind<ConstPartial<Constant>, Tag>

final class Const<Constant, Tag>: ConstOf<Constant, Tag> {
	let constant: Constant

	init(_ constant: Constant) {
		self.constant = constant
	}

	static func fix(
		_ fConstant: ConstOf<Constant, Tag>
	) -> Const<Constant, Tag> {
		fConstant as! Const<Constant, Tag>
	}
}

postfix operator ^
postfix func ^<Constant, Tag>(
	_ fConstant: ConstOf<Constant, Tag>
) -> Const<Constant, Tag> {
	Const.fix(fConstant)
}

protocol Functor {
	static func map<A, B>(
		_ fA: Kind<Self, A>,
		_ transform: (A) -> B
	) -> Kind<Self, B>
}

extension ConstPartial: Functor {
	static func map<OldTag, NewTag>(
		_ fA: ConstOf<Constant, OldTag>,
		_ transform: (OldTag) -> NewTag
	) -> ConstOf<Constant, NewTag> {
		Const(fA^.constant)
	}
}

func boolConst<A>(_ constant: Bool) -> Const<Bool, A> {
	Const(constant)
}
\end{minted}

\end{enumerate}

\section*{Question 3.}

\begin{enumerate}[label=(\alph*)]

\item For $\delta$ to define a natural transformation, the following diagram must commute ($\forall f: X \rightarrow Y$ in $\textrm{Hom}(X, Y), \textrm{ with } X, Y \in \textbf{Obj}(\textbf{Set})$).

(Note, we can drop the $\textrm{id}_{\textbf{Set}}$ prefixes along the top row for $\textrm{id}_{\textbf{Set}}(X)$,  $\textrm{id}_{\textbf{Set}}(f)$, and $\textrm{id}_{\textbf{Set}}(Y)$ since the identity functor maps all objects and morphisms to themselves.)

\[\xymatrixcolsep{4pc}\xymatrix{
X \ar[d]^{\delta_X} \ar[r]^f &Y\ar[d]^{\delta_Y}\\
X \times X \ar[r]^{\textrm{Double}(f)}          &Y \times Y}
\]

Following the bottom path for an arbitrary $x \in X$,

\[x \xmapsto{\delta_X} (x, x) \xmapsto{\textrm{Double}(f)} (f(x), f(x))\]

And the upper,

\[x \xmapsto{f} f(x) \xmapsto{\delta_Y} (f(x), f(x))\]

Both paths are equivalent! $\delta$ is indeed a natural transformation between $\textrm{id}_{\textbf{Set}}$ and $\textrm{Double}$.

\item

\begin{minted}{swift}
func diag<A>(_ a: A) -> (A, A) {
	(a, a)
}
\end{minted}

\end{enumerate}

\section*{Question 4.}

\begin{enumerate}[label=(\alph*)]

\item Since $t, t'$ are both terminal objects, each have unique maps, say $u \textrm{ and } v$, to one another.

\[\xymatrix{t \ar@{-->}@/^/[r]^{u} & t' \ar@{-->}@/^/[l]^{v}}\]

Moreover, since the containing category must contain composite morphisms, $\textrm{Hom}(t, t)$ and $\textrm{Hom}(t', t')$ must contain $v \circ u$ and $u \circ v$, respectively. But, again leaning on their terminality, there can only be one endomorphism for each—their identities. This forces both compositions to cancel one another out, making $t \cong t'$.

\item Mirroring the approach in (a), we lean on the product’s universal properties.

That is, there exists unique morphisms between $p, p'$ that we’ll call $u$ and $v$.

\[
\xymatrix
@C=1pc
{
	  & p' \ar@{-->}@/^/[d]^{u} \ar@/_1.5pc/[ddl]_{\pi_a'} \ar@/^1.5pc/[ddr]^{\pi_b'} \\
	  & p \ar@{-->}@/^/[u]^{v} \ar[dl]_{\pi_a} \ar[dr]^{\pi_b} & \\
	a & & b
}
\].

Their composites must also be in the containing category, and again leaning on their universal properties, their must only be one endomorphism for each, that’re forced to be identities.

\item It could! Each of these isomprhisms are forced by the \textit{uniqueness} of morphisms originating from or pointing towards the universal constructions at hand.

\end{enumerate}

\section*{Question 5.}

\section*{Question 6.}

\section*{Question 7.}

\section*{Question 8.}

\begin{minted}{swift}
enum Either<Left, Right> {
	case left(Left)
	case right(Right)
}

func bimap<Left, Right, NewLeft, NewRight>(
	_ leftTransform: @escaping (Left) -> NewLeft,
	_ rightTransform: @escaping (Right) -> NewRight
)
-> (Either<Left, Right>)
-> Either<NewLeft, NewRight> {
	{ either in
		switch either {
		case let .left(left):
			return .left(leftTransform(left))
		case let .right(right):
			return .right(rightTransform(right))
		}
	}
}
\end{minted}

\end{document}
